% *************** Definicje stylu dokumentu ***************

% *********************************************************************************
% W piku tym zdefiniowany jest wygl¹d dokumentu.
% Zmiany tutaj nie s¹ konieczne o ile nie zamierzasz zmieniaæ wygl¹du dokumentu.
% *********************************************************************************

% *************** Za³adowanie pakietów ***************
\usepackage{graphicx}
\usepackage{epsfig}
\usepackage{amsmath}
\usepackage{amssymb}
\usepackage{amsthm}
\usepackage{booktabs}
\usepackage{svg}
\usepackage{stmaryrd}
\usepackage{url}
\usepackage{longtable}
\usepackage[figuresright]{rotating}
\usepackage{footmisc}
\usepackage{caption}

\usepackage{polski}
\usepackage[utf8]{inputenc}

\usepackage{indentfirst}

%\usepackage{ulem}

\usepackage{eso-pic}% http://ctan.org/pkg/eso-pic


\usepackage[ruled,vlined]{algorithm2e}
\usepackage{algorithmicx}

\usepackage{listings}

\captionsetup[table]{position=above}

%\usepackage[italictitle=true]{biblatex}


\renewcommand\familydefault{\sfdefault}


% *************** W³¹czenie tworzenia skorowidza ***************
%\makeindex

% *************** Dodanie do pozycji bibliograficznych informacji o numerze strony, na której jest ona cytowana ***************
\usepackage{citeref}
\renewcommand{\bibitempages}[1]{\newblock {\scriptsize [\mbox{str.\ }#1]}}

\renewcommand{\emph}[1]{\textit{#1}}

% *************** Definicje niektórych kolorów ***************
\usepackage{color}
\setlength{\skip\footins}{20pt}

\definecolor{greenyellow}   {cmyk}{0.15, 0   , 0.69, 0   }
\definecolor{yellow}        {cmyk}{0   , 0   , 1   , 0   }
\definecolor{goldenrod}     {cmyk}{0   , 0.10, 0.84, 0   }
\definecolor{dandelion}     {cmyk}{0   , 0.29, 0.84, 0   }
\definecolor{apricot}       {cmyk}{0   , 0.32, 0.52, 0   }
\definecolor{peach}         {cmyk}{0   , 0.50, 0.70, 0   }
\definecolor{melon}         {cmyk}{0   , 0.46, 0.50, 0   }
\definecolor{yelloworange}  {cmyk}{0   , 0.42, 1   , 0   }
\definecolor{orange}        {cmyk}{0   , 0.61, 0.87, 0   }
\definecolor{burntorange}   {cmyk}{0   , 0.51, 1   , 0   }
\definecolor{bittersweet}   {cmyk}{0   , 0.75, 1   , 0.24}
\definecolor{redorange}     {cmyk}{0   , 0.77, 0.87, 0   }
\definecolor{mahogany}      {cmyk}{0   , 0.85, 0.87, 0.35}
\definecolor{maroon}        {cmyk}{0   , 0.87, 0.68, 0.32}
\definecolor{brickred}      {cmyk}{0   , 0.89, 0.94, 0.28}
\definecolor{red}           {cmyk}{0   , 1   , 1   , 0   }
\definecolor{orangered}     {cmyk}{0   , 1   , 0.50, 0   }
\definecolor{rubinered}     {cmyk}{0   , 1   , 0.13, 0   }
\definecolor{wildstrawberry}{cmyk}{0   , 0.96, 0.39, 0   }
\definecolor{salmon}        {cmyk}{0   , 0.53, 0.38, 0   }
\definecolor{carnationpink} {cmyk}{0   , 0.63, 0   , 0   }
\definecolor{magenta}       {cmyk}{0   , 1   , 0   , 0   }
\definecolor{violetred}     {cmyk}{0   , 0.81, 0   , 0   }
\definecolor{rhodamine}     {cmyk}{0   , 0.82, 0   , 0   }
\definecolor{mulberry}      {cmyk}{0.34, 0.90, 0   , 0.02}
\definecolor{redviolet}     {cmyk}{0.07, 0.90, 0   , 0.34}
\definecolor{fuchsia}       {cmyk}{0.47, 0.91, 0   , 0.08}
\definecolor{lavender}      {cmyk}{0   , 0.48, 0   , 0   }
\definecolor{thistle}       {cmyk}{0.12, 0.59, 0   , 0   }
\definecolor{orchid}        {cmyk}{0.32, 0.64, 0   , 0   }
\definecolor{darkorchid}    {cmyk}{0.40, 0.80, 0.20, 0   }
\definecolor{purple}        {cmyk}{0.45, 0.86, 0   , 0   }
\definecolor{plum}          {cmyk}{0.50, 1   , 0   , 0   }
\definecolor{violet}        {cmyk}{0.79, 0.88, 0   , 0   }
\definecolor{royalpurple}   {cmyk}{0.75, 0.90, 0   , 0   }
\definecolor{blueviolet}    {cmyk}{0.86, 0.91, 0   , 0.04}
\definecolor{periwinkle}    {cmyk}{0.57, 0.55, 0   , 0   }
\definecolor{cadetblue}     {cmyk}{0.62, 0.57, 0.23, 0   }
\definecolor{cornflowerblue}{cmyk}{0.65, 0.13, 0   , 0   }
\definecolor{midnightblue}  {cmyk}{0.98, 0.13, 0   , 0.43}
\definecolor{navyblue}      {cmyk}{0.94, 0.54, 0   , 0   }
\definecolor{royalblue}     {cmyk}{1   , 0.50, 0   , 0   }
\definecolor{blue}          {cmyk}{1   , 1   , 0   , 0   }
\definecolor{cerulean}      {cmyk}{0.94, 0.11, 0   , 0   }
\definecolor{cyan}          {cmyk}{1   , 0   , 0   , 0   }
\definecolor{processblue}   {cmyk}{0.96, 0   , 0   , 0   }
\definecolor{skyblue}       {cmyk}{0.62, 0   , 0.12, 0   }
\definecolor{turquoise}     {cmyk}{0.85, 0   , 0.20, 0   }
\definecolor{tealblue}      {cmyk}{0.86, 0   , 0.34, 0.02}
\definecolor{aquamarine}    {cmyk}{0.82, 0   , 0.30, 0   }
\definecolor{bluegreen}     {cmyk}{0.85, 0   , 0.33, 0   }
\definecolor{emerald}       {cmyk}{1   , 0   , 0.50, 0   }
\definecolor{junglegreen}   {cmyk}{0.99, 0   , 0.52, 0   }
\definecolor{seagreen}      {cmyk}{0.69, 0   , 0.50, 0   }
\definecolor{green}         {cmyk}{1   , 0   , 1   , 0   }
\definecolor{forestgreen}   {cmyk}{0.91, 0   , 0.88, 0.12}
\definecolor{pinegreen}     {cmyk}{0.92, 0   , 0.59, 0.25}
\definecolor{limegreen}     {cmyk}{0.50, 0   , 1   , 0   }
\definecolor{yellowgreen}   {cmyk}{0.44, 0   , 0.74, 0   }
\definecolor{springgreen}   {cmyk}{0.26, 0   , 0.76, 0   }
\definecolor{olivegreen}    {cmyk}{0.64, 0   , 0.95, 0.40}
\definecolor{rawsienna}     {cmyk}{0   , 0.72, 1   , 0.45}
\definecolor{sepia}         {cmyk}{0   , 0.83, 1   , 0.70}
\definecolor{brown}         {cmyk}{0   , 0.81, 1   , 0.60}
\definecolor{tan}           {cmyk}{0.14, 0.42, 0.56, 0   }
\definecolor{gray}          {cmyk}{0   , 0   , 0   , 0.50}
\definecolor{black}         {cmyk}{0   , 0   , 0   , 1   }
\definecolor{white}         {cmyk}{0   , 0   , 0   , 0   } 
\definecolor{mygray}        {cmyk}{0.03, 0.03, 0.03, 0.03}
% *************** W³¹czenie hyperlinków w dokumentach PDF ***************






\lstset{
	basicstyle=\footnotesize\ttfamily, % zmniejszenie czcionki
	keywordstyle=\color{blue},
	stringstyle=\color{violetred},
	commentstyle=\color{green}, 
	frame=single,
	frameround=tttt,
	framesep=5pt,
}



\ifpdf
    \pdfcompresslevel=9
        \usepackage[plainpages=false,pdfpagelabels,bookmarksnumbered,%
        colorlinks= true,%
        linkcolor=black, % blue,%
        citecolor=black, %sepia,%
        filecolor=black, %maroon,%
        pagecolor=black, %red,%
        urlcolor=black, %sepia,%
        pdftex,%
        unicode]{hyperref} 
%    \input supp-mis.tex
%    \input supp-pdf.tex
    \pdfimageresolution=600
    \usepackage{thumbpdf} 
\else
    \usepackage{hyperref}
\fi

\hypersetup{
    colorlinks=true,     % Aktywuje kolorowanie (wymagane do podkreślenia linków)
    linkcolor=black,     % Ustawia kolor na czarny
    urlcolor=black,      % Ustawia kolor na czarny
    pdfborder={0 0 1}    % Dodaje podkreślenie linków (1 - grubość linii)
}

\usepackage{memhfixc}

% *************** Wygl¹d strony ***************
%\settypeblocksize{*}{36pc}{1.618}
\settypeblocksize{23cm}{16cm}{*}

\setlrmargins{*}{2cm}{*} % wielkoϾ prawego marginesu
\setulmargins{*}{*}{1.3}

%\special{papersize=210mm,297mm}
%\usepackage{geometry} 
%\newgeometry{tmargin=2cm, bmargin=2cm, lmargin=2.5cm, rmargin=2cm} 


\setheadfoot{\onelineskip}{2\onelineskip}
\setheaderspaces{*}{2\onelineskip}{*}

\def\baselinestretch{1.3}

\checkandfixthelayout

% *************** Stylu rozdzia³ów i podrozdzia³ów ***************
\makechapterstyle{mychapterstyle}{%
    \renewcommand{\chapnamefont}{\LARGE\sffamily\bfseries}%
    \renewcommand{\chapnumfont}{\LARGE\sffamily\bfseries}%
    \renewcommand{\chaptitlefont}{\Huge\sffamily\bfseries}%
    \renewcommand{\printchaptertitle}[1]{%
        \chaptitlefont\hrule height 0.5pt \vspace{1em}%
        {##1}\vspace{1em}\hrule height 0.5pt%
        }%
    \renewcommand{\printchapternum}{%
        \chapnumfont\thechapter%
        }%
}

\chapterstyle{mychapterstyle}

\setsecheadstyle{\Large\sffamily\bfseries}
\setsubsecheadstyle{\large\sffamily\bfseries}
\setsubsubsecheadstyle{\normalfont\sffamily\bfseries}
\setparaheadstyle{\normalfont\sffamily}

\makeevenhead{headings}{\thepage}{}{\small\slshape\leftmark}
\makeoddhead{headings}{\small\slshape\rightmark}{}{\thepage}

% *************** Styl spisu treœci ***************
\settocdepth{subsection}

\setsecnumdepth{subsection}
\maxsecnumdepth{subsection}
\settocdepth{subsection}
\maxtocdepth{subsection}

% ********** Polecenia do mott **********
\setlength{\epigraphwidth}{0.57\textwidth}
\setlength{\epigraphrule}{0pt}
\setlength{\beforeepigraphskip}{1\baselineskip}
\setlength{\afterepigraphskip}{2\baselineskip}

\newcommand{\epitext}{\sffamily\itshape}
\newcommand{\epiauthor}{\sffamily\scshape ---~}
\newcommand{\epititle}{\sffamily\itshape}
\newcommand{\epidate}{\sffamily\scshape}
\newcommand{\episkip}{\medskip}

\newcommand{\myepigraph}[4]{%
	\epigraph{\epitext #1\episkip}{\epiauthor #2\\\epititle #3 \epidate(#4)}\noindent}

% *************** Inne ***************
\renewcommand{\thefootnote}{\fnsymbol{footnote}}




\renewcommand{\today}{\ifcase \month \or Styczeń\or Luty\or Marzec\or %
Kwiecień\or Maj \or Czerwiec\or Lipiec\or Sierpień\or Wrzesień\or Październik\or Listopad\or %
Grudzień\fi~ \number \year} 

\newcommand\mainmatterWithoutReset
{\edef\temppagenumber{\arabic{page}}%
	\mainmatter
	\setcounter{page}{\temppagenumber}%
}

\let\mainmatterorig\mainmatter
\renewcommand\mainmatter
{\edef\temppagenumber{\arabic{page}}%
	\mainmatterorig
	\setcounter{page}{\temppagenumber}%
}



\usepackage{pdfpages}

% Nowe liczniki dla każdego rodzaju figure
\newcounter{wykres}
\newcounter{kod}
\newcounter{rysunek}

% Definicje formatowania captionów dla poszczególnych typów figure
\renewcommand{\thewykres}{Wykres \arabic{wykres}}
\renewcommand{\thekod}{Kod \arabic{kod}}
\renewcommand{\therysunek}{Rysunek \arabic{rysunek}}

% Nowe środowiska floatujące z własnym formatowaniem captionów
\newenvironment{wykres}[1][]
{
    \refstepcounter{wykres} % Zwiększa licznik wykresu
    \begin{figure}[H] % Floatuje jak figure
    \captionsetup{type=figure, name=Wykres} % Ustawia typ figure i zmienia prefix na "Wykres"
    \caption{#1}
    \centering
}
{
    \end{figure}
}

\newenvironment{kod}[1][]
{
    \refstepcounter{kod} % Zwiększa licznik kodu
    \begin{figure}[H] % Floatuje jak figure
    \captionsetup{type=figure, name=Kod} % Ustawia typ figure i zmienia prefix na "Kod"
    \caption{#1}
    \centering
}
{
    \end{figure}
}

\newenvironment{rysunek}[1][]
{
    \refstepcounter{rysunek} % Zwiększa licznik rysunku
    \begin{figure}[H] % Floatuje jak figure
    \captionsetup{type=figure, name=Rysunek} % Ustawia typ figure i zmienia prefix na "Rysunek"
    \caption{#1}
    \centering
}
{
    \end{figure}
}




%\renewcommand\bibname{Literatura}


% *************** Koniec definicji stylu dokumentu ***************
