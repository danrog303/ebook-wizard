\chapter{Wstęp}
\label{sec:chapter1}

Światowy rynek e-booków w 2024 roku jest wyceniany na ponad 22,4 miliarda dolarów amerykańskich. Według analiz rynkowych przeprowadzonych przez przedsiębiorstwo Future Market Insights, do 2034 roku liczba ta ma osiągnąć wartość 36,2~miliarda USD~\cite{raport_rynku_ebookow_2024}. Czytelnictwo elektroniczne jest cenione przez czytelników z całego świata przede wszystkim ze względu na wygodę i ekologię. 

Wielu użytkowników korzysta z dedykowanych czytników e-booków do konsumowania treści cyfrowych. Inni czytelnicy, którzy nie chcą kupować osobnego urządzenia, czytają książki za pośrednictwem swoich komputerów, laptopów bądź tabletów.

Mając na uwadze raport przygotowany przez Bibliotekę Narodową~\cite{raport_czytelnictwa_2023}, e-booki raczej nieprędko wyprą standardowe, papierowe książki. Według tego raportu, w 2023 roku tylko 7\% ankietowanych czytelników zadeklarowało, że przeczytało co najmniej jedną książkę w formie elektronicznej.

Mimo to, warto mieć na uwadze, że odsetek ten cały czas rośnie. W roku 2022, czytanie e-booków zadeklarowało wyłącznie 5\% polskich czytelników. E-booki stają się bardziej popularne z każdym rokiem, a producenci czytników tacy jak Amazon Kindle lub Pocketbook prześcigają się, aby dostarczyć klientom jak najbardziej jakościowe produkty. 

Tworzy to szerokie możliwości dla twórców aplikacji, którzy mają szansę zaspokoić rosnące potrzeby użytkowników poprzez tworzenie aplikacji służących do czytania i katalogowania elektronicznych zbiorów.

Wspomnianą niszę stara się również zagospodarować aplikacja Ebook-Wizard będąca przedmiotem niniejszej pracy inżynierskiej. Będąc swego rodzaju dyskiem w chmurze, przeznaczonym dla czytelników literatury cyfrowej, wyżej wymieniona aplikacja stara się zapewnić użytkownikom zestaw narzędzi służących nie tylko do czytania, ale i tworzenia plików z elektronicznymi książkami.

\section{Cel pracy}
\label{sec:1:chapter1}
Celem pracy jest zaprojektowanie i zaimplementowanie aplikacji internetowej umożliwiającej tworzenie, edytowanie i konwertowanie e-booków.

Zakres pracy obejmuje zaprojektowanie aplikacji poprzez sporządzenie listy wymaganych funkcjonalności, wybranie języków programowania oraz bibliotek, zaimplementowanie części frontendowej oraz backendowej, konfigurację usług sieciowych wymaganych przez aplikację, testy aplikacji, a także opublikowanie aplikacji w Internecie.

Aplikacja ma cechować się możliwością dostępu do bazy e-booków z możliwie jak największej liczby urządzeń, takich jak komputery, smartfony oraz tablety. Aby osiągnąć ten cel, jako środowisko uruchomieniowe aplikacji została wybrana przeglądarka internetowa. Dodatkowo, interfejs graficzny aplikacji zostanie zaimplementowany w zgodzie z techniką RWD (\textit{Responsive Web Design}). 

Konkretne funkcjonalności, którymi cechować się ma omawiana aplikacja, zostaną szczegółowo omówione w dalszej części pracy, a zwłaszcza w rozdziale drugim.

\section{Omówienie zawartości pracy}
\label{sec:1:chapter2}

Niniejszy dokument składa się z sześciu rozdziałów. Pierwszy rozdział to wstęp zawierający wprowadzenie do tematyki książek elektronicznych oraz omówienie zawartości pracy. Drugi rozdział zawiera omówienie funkcji i cech aplikacji oraz przegląd konkurencyjnych rozwiązań. W rozdziale trzecim przedstawione są podstrony wchodzące w skład interfejsu graficznego aplikacji. Rozdział ten zawiera zrzuty ekranu prezentujące wygląd poszczególnych podstron. Czwarty rozdział omawia techniczne aspekty implementacji aplikacji. Jest to rodział, który prezentuje i omawia najważniejsze części kodu źródłowego budującego serwis Ebook-Wizard. W tym rozdziale omówione są również biblioteki, które zostały wykorzystane do budowy aplikacji. Rozdział piąty zawiera omówienie technik informatycznych, dzięki którym zapewnione jest bezpieczeństwo i stabilność działania aplikacji. Ostatni rozdział zawiera podsumowanie zrealizowanych działań oraz możliwe kierunki dalszego rozwoju aplikacji.
