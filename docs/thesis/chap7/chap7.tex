\chapter{Wnioski}
\label{sec:chapter7}

\section{Podsumowanie przeprowadzonych prac}

Wszystkie cele postawione w rozdziale pierwszym zostały osiągnięte. Aplikacja Ebook-Wizard została zaprojektowana w oparciu 
o analizę istniejących rozwiązań, a następnie zaimplementowana i opublikowana w Internecie. Aplikacja będąca efektem prac spełnia standardy bezpieczeństwa i posiada wszystkie zadeklarowane funkcjonalności. Dodatkowo, funkcjonalności te zostały przetestowane zarówno przy użyciu testów automatycznych, jak i testów manualnych konsultowanych z promotorem. Dostęp do aplikacji możliwy jest przy użyciu przeglądarki internetowej zarówno na komputerach, jak i smartfonach i tabletach.

Realizacja projektu okazała się być złożonym przedsięwzięciem łączącym ze sobą kilka różnych dziedzin: projektowanie interfejsów graficznych, planowanie architektury systemów informatycznych, programowanie, ale również analizę rynku e-booków oraz analizę potrzeb użytkowników. Proces implementacji wymagał elastycznego podejścia do pojawiających się zadań oraz ciągłego doskonalenia implementowanych rozwiązań. Pomimo wielu wyzwań, projekt zakończył się sukcesem, dzięki czemu aplikacja nie tylko spełnia wszystkie omówione założenia, ale także oferuje solidną bazę do dalszego rozwoju.

\section{Dalsze kroki rozwoju aplikacji}

Jako dalszy kierunek rozwoju aplikacji, można z pewnością wskazać dodanie opcji płatności. Użytkownicy mogliby za wykupienie płatnej subskrypcji otrzymać dostęp do większej ilości pamięci dyskowej lub odblokować dodatkowe funkcjonalności serwisu, niedostępne dla użytkowników korzystających z darmowej wersji serwisu. Ta opcja nie została dodana w trakcie prac nad projektem z powodów fiskalnych. Legalne rozliczanie dochodów z witryny internetowej wymaga powzięcia pewnych czynności prawnych. W początkowej fazie rozwoju projektu zdecydowano się na ograniczenie zobowiązań fiskalno-prawnych, i zadecydowano o odroczeniu tych czynności do czasu zebrania odpowiednio dużej bazy użytkowników.

Monetyzacja aplikacji pozwoliłaby na dalsze prace nad stabilnością aplikacji. Dzięki źródłu dochodów, możliwe byłoby wynajęcie zapasowego serwera, który obsługiwałby użytkowników w razie awarii serwera głównego.

Ponadto, do serwisu Ebook-Wizard planowane jest dodanie dwóch kolejnych modułów, modułu audiobooków oraz modułu komiksów. Moduł obsługi audiobooków pozwalałby użytkownikom na importowanie audiobooków z dysku, katalogowanie ich oraz słuchanie w aplikacji. Ta opcja nie została zaimplementowana w obecnej wersji aplikacji dla minimalizacji przestrzeni dyskowej zajmowanej przez użytkowników. Moduł komiksów pozwalałby użytkownikom na zarządzanie komiksami w formatach cbz, cbr, mobi oraz pdf.

Kolejnym z możliwych kierunków rozwoju jest rozbudowa funkcji społecznościowych serwisu, na przykład o opcję budowania listy znajomych i wzajemnego recenzowania książek.

Jak zaprezentowano w niniejszym rozdziale, aplikacja Ebook-Wizard posiada znaczący potencjał rozwojowy, co stwarza możliwość jej długoterminowego udoskonalania i rozwijania przez wiele lat.